% Template ripped from:
% http://www.cs.technion.ac.il/~yogi/Courses/CS-Scientific-Writing/examples/simple/simple.htm

\title{Differential Privacy with Dependent Types}
\author{
        Casper Holmgreen \\
        Department of Computer Science\\
        DIKU - Datalogisk Institut, Københavns Universitet\\
        \and
        Knut Liest\o l\\
        Department of Computer Science\\
        DIKU - Datalogisk Institut, Københavns Universitet\\
}
\date{\today}

\documentclass[12pt]{article}

\usepackage{listings}

\begin{document}
\maketitle

\lstset{language=Haskell}

\begin{abstract}
This is the paper's abstract \ldots
\end{abstract}

\section{Introduction}\label{sec:introduction}

% \subsection{Motivation/Overview}

With each passing day, more and more of our personal information is being collected, cataloged and analyzed by an ever increasing number of interested parties.
Their interests can range from targeted advertising to malicious and potentially illegal actions and everything in between.
We are the producers of this data, however, and we should be concerned with how it is being used.

For example, our medical records consist of very personal information which we reasonably expect to remain private.
Many countries have imposed regulations requiring a baseline of privacy for systems maintaining sensitive information, e.g, the Health Insurance Portability and Acountability Act (HIPAA) in the USA.
However, the aggregation and availability of an entire populations medical records would be a huge boon for medical researchers in need of statistical data.
And so we are faced with a classic balancing act: how do we balance individuals' privacy against the usefulness of a dataset?

Differential privacy\cite{journals/cacm/Dwork11} is an emerging field aiming to answer this question.
The central concept in differential privacy is indistinguishability, i.e, a query against a dataset should return more-or-less the same result regardless of whether or not a particular individuals records were included in the data.
If the results are indistinguishable, then the records of that particular individual must be unidentifiable.

Many metrics and algorithms have been developed by the differential privacy community.
Each algorithm is typically bundled with a formal proof that it meets some constraints or has some privacy-related properties.
The burden of producing such a proof and implementing the described algorithm is typically manual, and therefore error-prone.

PINQ\cite{conf/sigmod/McSherry09} is a differential privacy framework that takes a different approach.
It is a SQL-like query DSL for the .NET languages which guarantees differential privacy by construction.
Users of PINQ are able to compose carefully implemented primitives to build computations which run against raw data but cannot break differential privacy guarantees.
A protected runtime-system rejects queries whose cost exceed the remaining privacy budget.

These cost vs. budget checks are made at runtime.
An analyst has no way of knowing whether a query will be accepted by the runtime-system unless they are manually tracking their budget.
Reed and Pierce\cite{conf/icfp/ReedP10} go a step further and build a strongly-typed programming language which represents query costs in the types.
We aim to extend this idea by showing that a dependent type system is a natural fit for capturing differential privacy requirements.

Strongly-typed languages are capable of statically verifying programs for type-correctness, precluding many potential sources of runtime errors: ``well-typed programs can't go wrong''.
We extend this notion of being ``well-typed'' to include differential privacy metrics.
Just as PINQ is embedded in the .NET languages (particularly C\#), we plan to embed our implementation within the dependently-typed, functional programming language: Idris \footnote{http://idris-lang.org}.
This allows us to take advantage of Idris' parser and advanced type checker and all of the improvements being made to them by the open-source community.

\begin{lstlisting}
    data SensitiveFunction : Sensitivity -> Type where
      MkSensitiveFunction : (a -> b) -> SensitiveFunction s
\end{lstlisting}

Well-typed programs in our embedded language can't go wrong and also can't violate their privacy requirements.
Formal proofs for type-correct algorithms written in our language are unnecessary - the program is the proof!
Thus, all type-correct programs must respect expected privacy requirements.

\subsection{Related work}

\begin{itemize}
  \item PINQ
  \item (D)Fuzz
  \item Airavat
  \item \ldots
\end{itemize}

\subsection{Contributions}

\paragraph{Outline}
The remainder of this article is organized as follows.
Section~\ref{previous work} gives account of previous work.
Our new and exciting results are described in Section~\ref{results}.
Finally, Section~\ref{conclusions} gives the conclusions.

\section{Background}\label{sec:background}

\begin{itemize}
  \item Relational algebra
  \item Diff.Priv.
  \item Dep.Types
\end{itemize}

\section{Implementation}\label{sec:implementation}

\section{Evaluation / Validation}\label{sec:evaluation}

\section{Discussion}\label{sec:discussion}

\subsection{Future work}

\section{Conclusions}\label{sec:conclusions}
We worked hard, and achieved very little.

\bibliographystyle{abbrv}
\bibliography{main}

\end{document}
