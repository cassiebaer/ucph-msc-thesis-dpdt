% Template ripped from:
% http://www.cs.technion.ac.il/~yogi/Courses/CS-Scientific-Writing/examples/simple/simple.htm

\title{Differential Privacy with Dependent Types}
\author{
        Casper Holmgreen \\
        Department of Computer Science\\
        DIKU - Datalogisk Institut, K\o benhavns Universitet\\
        \and
        Knut Liest\o l\\
        Department of Computer Science\\
        DIKU - Datalogisk Institut, K\o benhavns Universitet\\
}
\date{\today}

\documentclass[12pt]{article}

\usepackage{graphicx}
\usepackage{listings}
\usepackage{amsfonts,amsthm,amsmath}

\newtheorem{defn}{Definition}[section]

\begin{document}
\maketitle

\lstset{language=Haskell,basicstyle=\footnotesize,frame=single,
        numbers=left}

\begin{abstract}
This is the paper's abstract \ldots
\end{abstract}

\section{Introduction}\label{sec:introduction}

With each passing day, more and more of our personal information is being collected, cataloged and analyzed by an ever increasing number of interested parties.
Their interests can range from targeted advertising to malicious and potentially illegal actions and everything in between.
As the producers of this data, we should be concerned with how it is being used.

[maybe copy paste the rest of this from the original?]

\section{Dependent Types in Idris}\label{sec:dependent_types_in_idris}

Concepts used later in the paper that need to be defined here:
\begin{itemize}
\item TBD
\end{itemize}

\section{Differential Privacy}\label{sec:differential_privacy}

In this section, we describe the central ideas and metrics behind differential privacy.

\section{Function Sensitivity}\label{sec:function_sensitivity}

Function sensitivity plays a central role in much of the differential privacy literature.
We can make assertions about an entire program if we understand the sensitivities of the functions with which it is composed.
In this section, we describe function sensitivity and then demonstrate how we use dependent types to model it in Idris.

Function sensitivity captures the idea of how relatively ``far'' a function can magnify the distance between pairs of inputs.
We say that a function is $c$-sensitive if, for all pairs of inputs, the distance between the outputs is not $c$ times greater than the original distance between the inputs.

	$$ \forall x,y. d(f(x),f(y)) \le c \times d(x,y) $$

For example, imagine a function $f : \mathbb R \rightarrow \mathbb R$ and Euclidian distance function $d_{\mathbb R}(x,y) = |x-y|$.
Now sample two random values from a 1D line: $x,y$.

Consider the case where $f(x) = \texttt{add10}(x) = x + 10$.
Clearly, it doesn't matter which $x$ and $y$ you sample; the distance $d_{\mathbb R}(x,y)$ will always equal $d_{\mathbb R}(f(x),f(y))$.
We say that \texttt{add10} is a 1-sensitive function.
Now consider the case where $f(x) = \texttt{doubleThenAddTen}(x) = 2x + 10$.
Distances can be now be doubled (but no more), so it is a 2-sensitive function.
Intuitively, when dealing with linear functions and the Euclidian distance function, the largest linear coefficient will dictate the c-sensitivity.
Higher order polynomials can be only be described as being $\infty$-sensitive.

Linear functions and Euclidian distances are not terribly interesting.
Function sensitivity applies to many other interesting domains such as databases containing private tax records or health information.

We will use $\mathbb D$ to describe the domain of databases.
Hence, a function $f : \mathbb D \rightarrow \mathbb D$ is a function across databases.
Our distance function will just be the symmetric difference of two databases; i.e. $d_{\mathbb D}(D_1,D_2) = | D_1 \oplus D_2 |$.
There is a common case in the differential privacy literature where two databases differ by exactly 1 row (i.e. $d_{\mathbb D}(D_1,D_2)=1$).
This is notationally represented by $D_1 \sim D_2$.

Function sensitivies compose intuitively, too.
The composition of \texttt{add10 . doubleThenAdd10} is 2-sensitive.
\texttt{doubleThenAdd10 . doubleThenAdd10} is 4-sensitive.
Function sensitivity represents the maximum multiplicative factor by which the distances increase.
So the composition of two functions, $f$ and $g$, which are $c$- and $c'$-sensitive, respectively, intuitively is going to be $(c \times c')$-sensitive.

% TODO : discuss any function that is c-sensitive is also c'-sensitive, for all $c < c'$.

\section{Typing the Relational Algebra}\label{sec:typing_the_relational_algebra}

In this section, we review the relational algebra and show how it can be modelled in a dependently typed language.

\section{Our implementation}\label{sec:our_implementation}

In this section, we bring together the material of the previous chapters and outline our entire implementation.

\section{Evaluation}\label{sec:evaluation}

In this section, we evaluate our language and discuss various things about it.

\section{Conclusion}\label{sec:conclusion}

We worked very hard, but achieved very little.

\bibliographystyle{abbrv}
\bibliography{main}

\end{document}
